%%%%%%%%%%%%%%%%%%%%%%%%%%%%%%%%%%%%%%%%%%%%%%%%%%%%%%%%%%%%%%%%
% %
% Kevin Russell %
% ECE 351-51 %
% Lab 9 %
% October 27, 2020 %
% %
% %
%%%%%%%%%%%%%%%%%%%%%%%%%%%%%%%%%%%%%%%%%%%%%%%%%%%%%%%%%%%%%%%%
\documentclass[12pt, titlepage]{article}

\usepackage[margin=1in]{geometry}
\usepackage[strict]{changepage}
\usepackage{float}
\usepackage{fancyhdr}
\usepackage{mhchem}
\usepackage{siunitx}
\usepackage{wrapfig, booktabs}
\usepackage{enumitem}
\usepackage{caption}
\usepackage{commath}
\usepackage{amsmath}
\usepackage[hang]{footmisc}
\usepackage{multicol}
\usepackage{amsfonts}
\usepackage{mathrsfs}
\usepackage{graphics}
\usepackage{graphicx}
\usepackage{listings}


\newcommand{\experimentDate}{October 27, 2020}
\newcommand{\className}{ECE 351}
\newcommand{\sectionNumber}{51}
\newcommand{\experimentNumber}{LAB 9}
\author{Kevin Russell}
\newcommand{\authorLastName}{Russell}
\title{Fast Fourier Transform}
\newcommand{\experimentShortName}{FFT}

\date{\parbox{\linewidth}{\centering%
  \experimentDate
  \endgraf\bigskip
  \className\ -- Section \sectionNumber\
}}

\pagestyle{fancy}
\fancyhf{}
\rhead{\authorLastName\ \thepage}
\lhead{\experimentShortName}
\cfoot{\className\ -- \experimentNumber}

\usepackage{color}
\usepackage{sectsty}

\definecolor{WordSectionBlue}{RGB}{30, 90, 147}

\allsectionsfont{\color{WordSectionBlue}}

\newcommand{\gpmol}{\si{\gram\per\mol}}
\renewcommand{\baselinestretch}{2.0}
\setlength{\parindent}{0em}
\setlength{\parskip}{1em}





\begin{document}

 \newpage
	    \maketitle
\newpage
    \tableofcontents
    
\newpage
    \section{Introduction}    
    The purpose of this lab was to become familiar with the Fast Fourier Transform in Python.  This version of the Fourier Transform is typically used on spectrum analyzer, so it will be helpful to understand how it works before entering the a job as and Electrical Engineer.
    
    \section{Equations}
    
    The following are the signals used in parts 1 through 4 of this lab
    \begin{equation}
        \cos{(2\pi t)}
    \end{equation}
    \begin{equation}
        5\sin{(2\pi t)}
    \end{equation}
    \begin{equation}
        2\cos{((2\pi t)-2)}+\sin^2{((2\pi 6t)+3)}
    \end{equation}
    
    
    The following equations were found as part of the preliminary for lab 6.  They will be used for task 5 of this lab.
    
     \begin{equation}
           a_0=0
           \nonumber
       \end{equation}
       \begin{equation}
           a_K=0
           \nonumber
       \end{equation}
       \begin{equation}
           b_K=\frac{2-2\cos{(\pi K)}}{\pi K}
           \nonumber
       \end{equation}
       \begin{equation}
           x(t)=\sum_{K=1}^{\infty}\frac{2-2\cos{(\pi K)}}{\pi K}\sin{(\frac{2\pi Kt}{T})}
       \end{equation}
    
    \section{Methodology}
    To perform this lab, the Fast Fourier Transform (FFT) function was defined as a user defined function.  The code for the FFT was given in the lab handout and needed to be inserted into a user defined function in order to operate properly.  This function took in the input signal and the sampling frequency, outputting the FFT magnitude, phase, and new frequency to properly graph the function.
    
    For each task for tasks 1 through 3, the signal from equations 1 through 3 were respectively defined and then run through the FFT function.  The input frequency of each was set to 100Hz.  Each function was then plotted from 0 to 2s by plotting a subplot of the original time domain function, the full spectrum frequency magnitude, full spectrum frequency angle, and then a focused frequency magnitude and phase plot looking more in-depth at the important frequencies.  
    
    Task 4 repeated the same process, however a new FFT function was defined that applied a filter to the phase output.  This filtered out and phase shift shown for frequency magnitudes less that 1e-10, setting the phase to zero.  To complete this, a for loop was defined to do this, inside of the new FFT user defined function.  Tasks 1 through 3 were then redone with this new FFT function.
    
    Finally, the fourier series approximation from lab 6 was also run through the new FFT funtion, plotting the output in the same manner as described for tasks 1 through 4.  The only difference is the plot went from 0 to 16s.
   
    \section{Results}
    The following figure is the original FFT output of equation 1.  It can be seen that the phase is very noisy for all part is the spectrum and does not clearly show the phase shift for the two frequencies that are most prominent on the magnitude plots.  Additionally, when limiting the x-axis frequency, it can be seen that the -1 and 1Hz frequencies are output of this FFT.
   
        \begin{figure}[h!]
            \centering
            \includegraphics[scale = .5]{Figure 2020-10-27 205305.png}
            \caption{Equation 1, Not Clean}
            \label{fig:my_label}
        \end{figure}
    
    \clearpage
    The next figure is the original FFT output of equation 2.  It can be seen that the phase is very noisy for all part is the spectrum and does not clearly show the phase shift for the two frequencies that are most prominent on the magnitude plots, as with the first plot.  Additionally, when limiting the x-axis frequency, it can be seen that the -1 and 1Hz frequencies are output of this FFT.
    
    \begin{figure}[h!]
        \centering
        \includegraphics[scale = .5]{Figure 2020-10-27 205305 (1).png}
        \caption{Equation 2, Not Clean}
        \label{fig:my_label}
    \end{figure}
    
    \clearpage
    Figure 3 is the original FFT output of equation 3.  It can be seen that the phase is also very noisy for all part is the spectrum and does not clearly show the phase shift for the two frequencies that are most prominent on the magnitude plots.  Additionally, when limiting the x-axis frequency, it can be seen that there are 5 different frequencies are output of this FFT, approximately -13,-3,0,3, and 13Hz.
    
    \begin{figure}[h!]
        \centering
        \includegraphics[scale = .5]{Figure 2020-10-27 205305 (2).png}
        \caption{Equation 3, Not Clean}
        \label{fig:my_label}
    \end{figure}
    
    
    \clearpage
    Next, the same three signals were re-plotted using the new FFT function that filters out all the small phase magnitudes.  Figures 4 and 5 show that both equations 1 and 2 have the same phase angles of 1.5 and -1.5 for -1 and 1Hz, respectively.   Figure 6 shows that there are only two phase angles for equation 3, being at -3 and 3 for the -13 and 13Hz, respectively.  From these plots, it can be seen that the new FFT significantly filters the phase plot for all the small magnitudes so it can now be interpreted and properly read for analysis of the signal.
    
     \begin{figure}[h!]
        \centering
        \includegraphics[scale = .5]{Figure 2020-10-27 205305 (3).png}
        \caption{Equation 1, Clean}
        \label{fig:my_label}
    \end{figure}
    
     \begin{figure}[h!]
        \centering
        \includegraphics[scale = .5]{Figure 2020-10-27 205305 (4).png}
        \caption{Equation 2, Clean}
        \label{fig:my_label}
    \end{figure}
    
     \begin{figure}[h!]
        \centering
        \includegraphics[scale = .5]{Figure 2020-10-27 205305 (5).png}
        \caption{Equation 3, Clean}
        \label{fig:my_label}
    \end{figure}
    
    \clearpage
    Finally, figure 7 shows the new FFT output for the Fourier Series signal from lab 6, outlined in equation 4.  As with the previous three plots, this plot clearly defines the significant frequencies attributed with the square wave as well as the correlated significant phase angles.
    
     \begin{figure}[h!]
        \centering
        \includegraphics[scale = .5]{Figure 2020-10-27 205305 (6).png}
        \caption{Equation 4, Clean}
        \label{fig:my_label}
    \end{figure}
    \clearpage
   
    
    \section{Questions}
    \begin{enumerate}
        \item What happens if fs is lower? If it is higher?
        
        Changing the frequency only alters the range of data that is being looked at.  The lower the frequency, the less the spread of data.  Increasing the frequency will have a larger range of data shown.  This comes through the nature of what frequency is doing in this operation.  The frequency is the number of samples every second.  Increasing the frequency allows for more samples to be taken, therefore looking at a larger set of data.  Decreasing the frequency limits the number of samples taken, reducing the size of the data set.  This is demonstrated by the following three figures which show the variation of frequency at 50, 100, and 1000Hz.
        
        \begin{figure}[h!]
            \centering
            \includegraphics[scale=.25]{Figure 2020-10-27 211319.png}
            \caption{50hz}
            \label{50Hz}
        \end{figure}
        \begin{figure}[h!]
            \centering
            \includegraphics[scale=.25]{Figure 2020-10-27 211319 (1).png}
            \caption{100hz}
            \label{50Hz}
        \end{figure}
        \begin{figure}[h!]
            \centering
            \includegraphics[scale=.25]{Figure 2020-10-27 211319 (2).png}
            \caption{1000hz}
            \label{50Hz}
        \end{figure}
        \clearpage
        \item What difference does eliminating the small phase magnitudes make?
        
        Eliminating the small phase magnitudes cleans up and filters out the plot of all the insignificant phases.  Since we only need to focus on the phase of the significant frequencies, all other phases can be filtered out and ignored.
        
        \item Verify the results from Tasks 1 and 2 using the Fourier transforms of cosine and sine in terms of Hz.  Explain the results.
        
        When taking the Fourier Transform of the cosine signal, the result is the following signal:
        
        \begin{equation}
            \nonumber
            \frac{1}{2}(\delta(f-1)+\delta(f+1))
        \end{equation}
        
        If this were plotted, the magnitude and phase plots would be the same as the FFT would show because these are delta functions with magnitude of 1/2 and are vertical lines starting at zero, stopping at 1/2.
        
        Similarly, when taking the Fourier Transform of the sine signal, the result is the following signal:
        
        \begin{equation}
            \nonumber
            \frac{5}{j2}(\delta(f-1)-\delta(f+1))
        \end{equation}
        
        If this were plotted, the magnitude and phase plots would be the same as the FFT would show because these are delta functions with magnitude of 1/2 and are vertical lines starting at zero, stopping at 1/2.
        
        These derivations show the accuracy of the FFT and how it incorporates the Fourier Transform and techniques already learned in class.
        
        \item Leave any feedback on the clarity/usefulness of the purpose, deliverables, and expectations for this lab.
        
        This lab was very clear on what was expected. The instructions written in lab were well defined and aided in the success of this lab. The lab instructions also clearly defined the deliverables that were required to include in this report. Overall, no improvements should be made to this lab.
    \end{enumerate}
       
    \clearpage
    \section{Conclusion}
    This lab provided a useful demonstration of the Fast Fourier Transform and showed how it can be used to isolate the significant frequencies of a signal. Additionally, this lab also showed how the FFT itself provides a noisy phase plot. By incorporating a for-loop filter, the small phase magnitudes can be removed to isolate the phase shift of the significant frequencies. All ideas in this lab were presented in a way that was successful in solidifying these concepts.
    
    \section{GitHub Link}
        https://github.com/russ1530

\end{document}
