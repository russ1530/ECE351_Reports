%%%%%%%%%%%%%%%%%%%%%%%%%%%%%%%%%%%%%%%%%%%%%%%%%%%%%%%%%%%%%%%%
% %
% Kevin Russell %
% ECE 351-51 %
% Lab 1 %
% September 1, 2020 %
% %
% %
%%%%%%%%%%%%%%%%%%%%%%%%%%%%%%%%%%%%%%%%%%%%%%%%%%%%%%%%%%%%%%%%
\documentclass[12pt, titlepage]{article}

\usepackage[margin=1in]{geometry}
\usepackage[strict]{changepage}
\usepackage{float}
\usepackage{fancyhdr}
\usepackage{mhchem}
\usepackage{siunitx}
\usepackage{wrapfig, booktabs}
\usepackage{enumitem}
\usepackage{caption}
\usepackage{commath}
\usepackage{amsmath}
\usepackage[hang]{footmisc}
\usepackage{multicol}
\usepackage{amsfonts}
\usepackage{mathrsfs}

\newcommand{\experimentDate}{September 1, 2020}
\newcommand{\className}{ECE 351}
\newcommand{\sectionNumber}{51}
\newcommand{\experimentNumber}{LAB 1}
\author{Kevin Russell}
\newcommand{\authorLastName}{Russell}
\title{Introduction to Python 3.x and Latex}
\newcommand{\experimentShortName}{Introduction}

\date{\parbox{\linewidth}{\centering%
  \experimentDate
  \endgraf\bigskip
  \className\ -- Section \sectionNumber\
}}

\pagestyle{fancy}
\fancyhf{}
\rhead{\authorLastName\ \thepage}
\lhead{\experimentShortName}
\cfoot{\className\ -- \experimentNumber}

\usepackage{color}
\usepackage{sectsty}

\definecolor{WordSectionBlue}{RGB}{30, 90, 147}

\allsectionsfont{\color{WordSectionBlue}}

\newcommand{\gpmol}{\si{\gram\per\mol}}
\renewcommand{\baselinestretch}{2.0}
\setlength{\parindent}{0em}
\setlength{\parskip}{1em}





\begin{document}

 \newpage
	    \maketitle
    
    \newpage
        \tableofcontents
    
    \newpage
        \section{Summary}
        \subsection{Part 1}
        From this part, I learned that there are several helpful keyboard shortcuts that make writing code in python much quicker.
        
        \subsection{Part 2}
        In part 2 of this lab, basic commands in python were demonstrated.  This included basic mathematical operations, printing outputs, using arrays, and making basic plots.  This will be helpful for guidance in future labs.
        
        \subsection{Part 3}
        In part 3 of this lab, a basic latex template was created.  This includes a common header, title page, table of contents, and the beginning of each content page.  A template that was used in previous courses was updated for this course.
        
    \newpage    
        \section{Questions}
        
        1.  I am most excited for ECE 320, Energy Systems, which I am currently enrolled in.  Power systems is very interesting to me, so I am looking forward to learning more about them.  So far, I have only taken circuits and digital logic, both of which I found very interesting and informative at different points in the semester.  Digital logic seemed for be more enjoyable because there were interesting labs with lots of problem solving.
        
        2.  For the overall lab, all expectations, instructions, and deliverable are clearly outlined.  I understand what is expected of me during this semester and what I will need to accomplish each week.
        
        Specifically for this weeks labs, some of the steps in Lab 0 were not very clear. Since I haven't used Spyder or Python before, I didn't know what to expect for outputs.  Although I was able to figure it out, it might be helpful to add images like in lab 1.  Lab 1 was straightforward and well written.  

	\section{GitHub Link}
        https://github.com/russ1530

\end{document}
